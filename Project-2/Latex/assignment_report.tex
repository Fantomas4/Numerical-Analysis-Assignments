% ------------------------------------------------------------------------------ %
% The ?rst information LATEX needs to know when processing an input ?le is
% the type of document the author wants to create. This is speci?ed with the
% \documentclass command.
%
%   \documentclass[options]{class}
%
% Here class speci?es the type of document to be created.

\documentclass[a4paper,11pt]{article}

% The above command instructs LATEX to typeset the document as an article with
% a base font size of eleven points, printing on A4 paper.
% ------------------------------------------------------------------------------- %


% ------------------------------------------------------------------------------- %
% While writing your document, you will probably ?nd that there are some
% areas where basic LATEX cannot solve your problem. If you want to include
% graphics, coloured text or source code from a ?le into your document, you
% need to enhance the capabilities of LATEX. Such enhancements are called
% packages. Packages are activated with the
%
%   \usepackage[options]{package}
%
% command, where package is the name of the package and options is a list of
% keywords that trigger special features in the package.
%
% ATTENTION: IN ORDER TO ENABLE THE GREEK TYPESETTING YOU NEED TO INCLUDE IN YOUR
%            PREAMBLE SPECIFIC PACKAGES VIA THE USE OF THE \usepackage COMMAND.
%
%            THE SAME IS TRUE FOR SPECIFIC MATH FONTS ETC.


% * Activating Greek fonts in Latex
  \usepackage[english,greek]{babel} % the last language is the default

  %% > UNCOMMENT if your editor uses iso-8859-7 encoding for Greek (typical in Windows System).
  %\usepackage[iso-8859-7]{inputenc}

  %% > UNCOMMENT if your editor uses Unicode encoding for Greek (typical in POSIX Systems).
  \usepackage[utf8x]{inputenc}

  % One bad thing about the babel package is that it cannot discriminate explicitly between
  % Greek and Latin fonts, so you have to state commands in order to signify where the
  % Latin charecters begin and where they end and the Greek characters begin. For this
  % job the \latintext and \greektext commands exist. However Latex give you the versatility
  % to create wildcards for all of each commands and thus, to create alias with shorter
  % word length. Below we create the aliaces \lt and \gt for the \textlatin and \textgreek
  % commands respectively.

  \newcommand{\lt}{\latintext}
  \newcommand{\gt}{\greektext}


% * Math packages
 % \usepackage{amsthm}
  \usepackage{amsmath}
 % \usepackage{amssymb}

% * graphics package
  \usepackage[pdftex]{graphicx} % remove the 'pdftex' option if not PDFLatex is used.

% * verbatim writing package (mainly used to import program's code)
  \usepackage{verbatim}

% Used to prevent errors while using >{\em} for array columns
  \usepackage{array}

% Used for the final array at exercise 4
  \usepackage{multirow}

% Used to get greek letter support in enumeration
  \usepackage{moreenum}

% Used for bold font in enumeration
  \usepackage{enumitem}

% ------------------------------------------------------------------------------- %


% ------------------------------------------------------------------------------- %
% Here we set the title, the author and the date of our document.
%
% * Setting the title of the document
  \title{1η Υποχρεωτική Εργασία \lt LaTeX} % Put your own title here

% * Setting the author or authors of the document
  \author{Ονοματεπώνυμο: Στέφανος Καραμπέρας  \\  ΑΕΜ: 2910}       % Put your own Name and AEM here

% * Setting the date of the document
  \date{\today}                                      % Put a specific date here
% ------------------------------------------------------------------------------- %


% =============================================================================== %
% ||                       HERE WE BEGIN OUR DOCUMENT                          || %
% =============================================================================== %
\begin{document}

% *** We are now inside the document everything from now on is VISIBLE!!! *** %

% Command that prints the title of your document
\maketitle

\section*{Άσκηση 1}
	\subsection*{Γενικές παρατηρήσεις}
	Ο κώδικας που έχει συνταχθεί για την άσκηση 1 βρίσκεται εντός του φακέλου {\lt ``Code/task-1" }.\\
	Οι ζητούμενοι αλγόριθμοι που εφαρμόζουν τις τεχνικές προσέγγισης ρίζας με τη μέθοδο της διχοτόμησης, τη μέθοδο {\lt Newton-Raphson} και τη μέθοδο της τέμνουσας, έχουν υλοποιηθεί
	εντός του αρχείου {\lt ``root\_estimation\_algorithms.py"}
	Επιπρόσθετα, εντός του αρχείου {\lt f\_function.py} έχουν δημιουργηθεί οι συναρτήσεις {\lt calculate\_f(x)}, {\lt calculate\_der\_1\_f(x)} και {\lt calculate\_der\_2\_f(x)}. Οι συναρτήσεις αυτές δέχονται ως όρισμα έναν
	αριθμό x και επιστρέφουν την ρίζα της {\lt $f(x)$}, {\lt $f'(x)$} και {\lt $f''(x)$} αντίστοιχα.
	Τέλος, με δεδομένο πως η ζητούμενη ακρίβεια ορίζεται στο 5ο δεκαδικό ψηφίο, η μεταβλητή {\lt t\_err}  στο αρχείο {\lt ``root\_estimation\_algorithms.py"} αρχικοποιείται με την τιμή {\lt t\_err} = 0.000005 (ακρίβεια 5ου δεκαδικού 		ψηφίου με στρογγυλοποίηση).
	\subsection*{Ερώτημα (α)}
		\paragraph{Αλγόριθμος για τη μέθοδο Διχοτόμησης}\mbox{} \\
			Η συνάρτηση που υλοποιήθηκε για την εκτέλεση του αλγορίθμου που εφαρμόζει τη μέθοδο της διχοτόμησης είναι η {\lt ``partitioning\_root\_estimation()"}. \\
			Η συνάρτηση αρχικά υπολογίζει το πλήθος των απαιτούμενων επαναλήψεων της μεθόδου διχοτόμησης για την επίτευξη της ζητούμενης ακρίβειας.  Στη συνέχεια, ξεκινά να εκτελεί το ορισμένο πλήθος επαναλήψεων, 					σταματώντας σε περίπτωση που βρει ακριβή ρίζα της {\lt$f(x)$} και τυπώντας τη ρίζα, ή συνεχίζοντας μέχρι να ολοκληρώσει το ορισμένο πλήθος επαναλήψεων, τυπώνοντας τελικά την προσέγγιση της ρίζας και το 						πλήθος επαναλήψεων που απαιτήθηκαν για την εκτίμησή της.
		\paragraph{Αλγόριθμος για τη μέθοδο {\lt Newton-Raphson }}\mbox{}\\
			Η συνάρτηση που υλοποιήθηκε για την εκτέλεση του αλγορίθμου που εφαρμόζει τη μέθοδο {\lt Newton-Raphson } είναι η {\lt `newton\_raphson\_root\_estimation()"}.\\
			Η συνάρτηση αρχικοποιεί το {$x_{n-1}$} χρησιμοποιώντας το πρώτο στοιχείο του πεδίου ορισμού [-2,2] της {\lt $f(x)$}, δηλαδή το -2. Ακολούθως, υπολογίζει διαδοχικά το  {$x_{n}$} μέχρις ότου
			η απόλυτη τιμή της διαφοράς {$x_{n-1}$}-{$x_{n-1}$} να είναι μικρότερη ή ίση του καθορισμένου αποδεκτού σφάλματος (ακρίβεια), δηλαδή $|${$x_{n}$} - {$x_{n-1}$}$|$ {$\leq$} {\lt t\_err}. Η συνάρτηση καταγράφει σε 					μια μεταβλητή {\lt iter\_count} το πλήθος επαναλήψεων που εκτελεί κατά την εκτέλεσή της. Με τη λήξη των επαναλήψεων λόγω επίτευξης της επιθυμητής ακρίβειας, τυπώνεται το πλήθος επαναλήψεων που 							εκτελέστηκαν και η προσέγγιση της ρίζας που υπολογίστηκε.
		\paragraph{Αλγόριθμος για τη μέθοδο Τέμνουσας}\mbox{} \\
			Η συνάρτηση που υλοποιήθηκε για την εκτέλεση του αλγορίθμου που εφαρμόζει τη μέθοδο της τέμνουσας είναι η {\lt ``secant\_root\_estimation()"}. \\
			Η συνάρτηση αρχικοποιεί το {$x_{n-1}$} χρησιμοποιώντας το πρώτο στοιχείο του πεδίου ορισμού [-2,2] της {\lt $f(x)$}, δηλαδή το -2. Επιπρόσθετα, αρχικοποιεί το {$x_{n}$} χρησιμοποιώντας το τελευταίο στοιχείο του 					πεδίου ορισμού [-2,2] της {\lt $f(x)$}, δηλαδή το 2.\\
			Ακολούθως, υπολογίζει διαδοχικά το {$x_{n+1}$} μέχρις ότου η απόλυτη τιμή της διαφοράς {$x_{n+1}$} - {$x_{n}$} να είναι μικρότερη ή ίση του καθορισμένου αποδεκτού σφάλματος (ακρίβεια), δηλαδή 
			$|${$x_{n+1}$} - {$x_{n}$}$|$ {$\leq$} {\lt t\_err}.
			
			
		

\section{Άσκηση 1 (Λύση)}
	
	\begin{center}
		\emph{{\tiny Α}{\scriptsize Β}{\footnotesize Γ}{\small Δ}{\normalsize Ε}{\large Ζ}{\Large Η}{\LARGE Θ}{\huge Ι}{\huge κ}{\LARGE λ}{\Large μ}{\large ν}{\normalsize ξ}{\small ο}{\footnotesize π}{\scriptsize ρ}{\tiny ς}}
	\end{center}

\vspace{20pt}

\section{Άσκηση 2 (Λύση)}

	\begin{center}	
		\lt
		\textit{Normal Italics \textbf{Bold\\}}
		{E}\textit{mphasized \underline{Underlined}}
	\end{center}

\section{Άσκηση 3 (Λύση)}
	
	\begin{equation*}
	a^2 + b^2 = c^2
	\end{equation*}
	\begin{equation*}
	e^{i\pi} = -1
	\end{equation*}
	\begin{equation*}
	\pi = \frac{c}{d}
	\end{equation*}
	\begin{equation*}
	\dfrac{d}{dx}\int_{a}^{x} f(s) ds = f(x)
	\end{equation*}
	\begin{equation*}
	f(x) = \sum_{i=0}^{\infty } {\dfrac{f^{(i)}(0)}{i!} x^i }
	\end{equation*}
	\begin{equation*}
	\textbf{\lt{Ax = b}}
	\end{equation*}
	\begin{equation*}
	||{x+y}||\leq||x||+||y||
	\end{equation*}
	
	\quad
	
	\begin{equation}
	\textbf{I} = 
	\begin{pmatrix}
	1&0&0&0 \\
	0&1&0&0 \\
	0&0&1&0 \\
	0&0&0&1 \\
	\end{pmatrix}
	\end{equation}
	
	\quad
	
	\begin{equation}
	\textbf{I} = 
	\begin{bmatrix}
	1&0&0&0 \\
	0&1&0&0 \\
	0&0&1&0 \\
	0&0&0&1 \\
	\end{bmatrix}
	\end{equation}
	
	\quad
	
	\begin{equation}
	\textbf{I} =
	\begin{Bmatrix} 
	1&0&0&0 \\
	0&1&0&0 \\
	0&0&1&0 \\
	0&0&0&1 \\
	\end{Bmatrix},\quad
	\textbf{I} =
	\begin{vmatrix} 
	1&0&0&0 \\
	0&1&0&0 \\
	0&0&1&0 \\
	0&0&0&1 \\
	\end{vmatrix},\quad
	\textbf{I} =
	\begin{Vmatrix} 
	1&0&0&0 \\
	0&1&0&0 \\
	0&0&1&0 \\
	0&0&0&1 \\
	\end{Vmatrix}
	\end{equation}

\vspace{20pt}

\section{Άσκηση 4 (Λύση)}
		
	\begin{table}[!h]
	\centering
	\begin{tabular}{ >{\em}l >{\em}c >{\em}c }
	Τέφας & 2 & 3 \\ 
	Πήτας & 5 & 6 \\ 
	Λάσκαρης & 8 & 9 \\ 
	\end{tabular}
	\end{table}	

	\begin{table}[!h]
	\centering
	\begin{tabular}{ | >{\em}l | >{\em}c | >{\em}c | }
	Κοτρόπουλος & 2 & 3 \\
	Πήτας & 5 & 6 \\ 
	Νικολαίδης & 8 & 9 \\ 
	\end{tabular}
	\end{table}		
	
	\begin{table}[!h]
	\centering
	\begin{tabular}{|>{\em}c|>{\em}c|>{\em}c|} \hline
	1 & 2 & 3 \\ \hline
	4 & 5 & 6 \\ \hline
	7 & 8 & 9 \\ \hline
	\end{tabular}
	\end{table}
	
	\begin{table}[!h]
	\centering
	\begin{tabular}{|>{\em}c|>{\em}c|>{\em}c|} \hline
	1 & 2 & 3 \\ \hline
	4 & 5 & 6 \\ \hline
	7 & 8 & 9 \\ \hline
	\end{tabular}
	\end{table}
	
	\begin{tabular}{ |>{\em}l|>{\em}l|>{\em}l| }
	\hline
	\multicolumn{3}{ |>{\em}c| }{Μέλη ΔΕΠ Πληροφορικής} \\
	\hline
	Λέκτορες & \lt VD & Δραζιώτης Κωνσταντίνος \\ \hline
	\multirow{2}{*}{Επίκουροι} 
          & \lt LN & Λάσκαρης Νικόλαος \\
	 & \lt TG & Τσουμάκας Γρηγόριος \\ \hline
	\multirow{3}{*}{Αναπληρωτές} 
	 & \lt TA & Τέφας Αναστάσιος \\
	 & \lt PN & Πλέρος Νίκος \\
	 & \lt PA & Παπαδόπουλος Απόστολος \\ \hline
	\multirow{3}{*}{Καθηγητές} 
	& \lt KC & Κοτρόπουλος Κωνσταντίνος \\
	& \lt PI & Πήτας Ιωάννης \\
	& \lt VI & Βλαχάβας Ιωάννης \\ 
	
	\hline
	\end{tabular}

\vspace{20pt}

\section{Άσκηση 5 (Λύση)}
		
	\begin{itemize}  
	\item \textit{Τέφας}
	\item \textit{Μπουζάς}
	\item \textit{Μπρούζα}
	\item \textit{Λάσκαρης}
	\item \textit{Κοτρόπουλος}
	\item \textit{Πήτας}
	\item \textit{Νικολαΐδης}	
	\end{itemize}	
	
	\begin{enumerate}  
	\item \textit{Τέφας}
	\item \textit{Μπουζάς}
	\item \textit{Μπρούζα}
	\item \textit{Λάσκαρης}
	\item \textit{Κοτρόπουλος}
	\item \textit{Πήτας}
	\item \textit{Νικολαΐδης}	
	\end{enumerate}	
	
	\begin{enumerate}[label=\textbf(\greek*)]
	\item \textit{Τέφας}
	\item \textit{Μπουζάς}
	\item \textit{Μπρούζα}
	\item \textit{Λάσκαρης}
	\item \textit{Κοτρόπουλος}
	\item \textit{Πήτας}
	\item \textit{Νικολαΐδης}	
	\end{enumerate}	


% =============================================================================== %
% ||                       HERE WE END OUR DOCUMENT                          || %
% =============================================================================== %
\end{document}
